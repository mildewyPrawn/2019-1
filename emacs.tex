% Lenguajes de Programación 2019-1
% Plantilla para reportes de laboratorio.

\documentclass[spanish,12pt,letterpaper]{article}

\usepackage[spanish]{babel}
\usepackage[utf8]{inputenc}
\usepackage{authblk}
\usepackage{listings}
\usepackage{dsfont}

\title{Comandos de emacs}
\author{Emiliano Galeana Araujo}
\affil{Facultad de Ciencias, UNAM}
\date{20 de Septiembre de 2018}

\begin{document}

\maketitle

\section{Comandos}
Lista de comandos que uso/he usado en emacs, espero les sirva y se puedan
acostombrar a ellos. 
\begin{itemize}
\item Para moverse en el texto
  \begin{itemize}
  \item \texttt{C-p} Sirve para posicionar el cursor una línea arriba.
  \item \texttt{C-n} Sirve para posicionar el cursor una línea abajo.
  \item \texttt{C-a} Sirve para posicionar el cursor al inicio de la línea.
  \item \texttt{C-e} Sirve para posicionar el cursos al final de la línea.
  \item \texttt{C-f} Sirve para avanzar un caracter, (Es análogo a usar la
    flecha $\rightarrow$).
  \item \texttt{C-b} Sirve para retroceder un caracter, (Es análogo a usar la
    flecha $\leftarrow$).
  \item \texttt{M-f} Sirve para avanzar una palabra.
  \item \texttt{M-b} Sirve para retroceder una palabra.
  \item \texttt{C-s} Abre un mini-buffer, escriben la palabra a buscar, y con
    \texttt{C-s} avanzan la búsqueda por palabra, con \texttt{enter} se
    posicionan en la palabra seleccionada y con el comando que cancela comandos
    cancelan la búsqueda.
  \item \texttt{M-v} Se mueve a la página anterior.
  \item \texttt{C-v} Se mueve a la página siguiente.
  \item \texttt{C-l} Mueve la página para que el cursor esté a la mitad de la
    pantalla, si se ejecuta de nuevo el comando, mueve la página para que esté
    al tope de la pantalla, y una vez más, lleva el cursor a lo más abajo de la
    pantalla, es cíclico.
  \end{itemize}
\item Para editar archivos
  \begin{itemize}
  \item \texttt{C-x C-s} Guarda las modificaciones en el archivo.
  \item \texttt{C-x C-f} Abrir un nuevo archivo, si existe lo abre, sino, lo
    crea.
  \item \texttt{C-x C-v} Creo que es análogo a \texttt{C-x C-f}.
  \item \texttt{C-y} Pegar (Es análogo a \texttt{C-v} En algún otro editor o en
    general en casi lo que sea.)
  \item \texttt{M-w} Copia (Es análogo a \texttt{C-c} En algún otro editor o en
    general en casi lo que sea), sirve sin seleccionar, pero no sé como, yo
    selecciono, y luego copio.
  \item \texttt{C-w} Corta (Es análogo a \texttt{C-x} En algún otro editor o en
    general en casi lo que sea), igual que \texttt{M-w} sirve sin seleccionar,
    pero no sé como, y mejor selecciono.
  \item \texttt{C-o} Es como dar enter, crea una línea debajo del cursor.
  \item \texttt{C-d} Es como borrar una línea, borra el salto de línea seguido
    del cursor, si no está al final de una línea, borra caracter por caracter.
    
  \end{itemize}
\item Seleccionar
  \begin{itemize}
  \item \texttt{C-space} Habilita un marcador en la posición del cursor, y con
    los comandos para moverse en el texto, se puede seleccionar parte de este,
    es útil para definir qué se quiere copiar, o borrar, se quita con el comando
    que cancela comandos, está al final, si se pone \texttt{C-space} de nuevo,
    se coloca un nuevo marcador.
  \item \texttt{C-x C-p} Me parece que es análogo a \texttt{C-x h}
  \item \texttt{C-x h} Selecciona todo el buffer.
  \item \texttt{M-h} Selecciona toda la línea y sin soltar \texttt{M}, por cada
    \texttt{h} que presionen, seleccionan una línea más, hacia abajo. 
  \end{itemize}
\item buffers
  \begin{itemize}
  \item \texttt{C-x k} Mata un buffer, esto es, lo cierra, si no se han guardado
    los cambios, pregunta si aún así se quiere cerrar.
  \item \texttt{C-x b} Lista los buffers abiertos, y escribiendo el nombre de uno,
    con \texttt{Tab} pueden autompletar, presionan \texttt{enter} y se mueven a
    ese buffer.
  \item \texttt{C-x $\leftarrow$} Se mueven al buffer a la izquierda.
  \item \texttt{C-x $\rightarrow$} Se mueven al buffer a la derecha.
  \item \texttt{C-x 2} Clona el buffer a la mitad de la pantalla horizontalmente.
  \item \texttt{C-x 3} Clona el buffer a la mitad de la pantalla verticalmente.
  \item \texttt{C-x 0} Borra la ventana en la que esté posicionado el cursor.
  \item \texttt{C-x 1} Borra todas las ventanas y deja solo en la que está
    posicionado el cursor.
  \item \texttt{C-x o} Se pueden mover entre las ventanas creadas.
  \end{itemize}
\item otros
  \begin{itemize}
  \item \texttt{C-x g} Comando que cancela comandos.
  \item \texttt{M-x tetris} Abre el juego de tetris :).
  \item \texttt{M-\%} Eso es \texttt{M-Shift-5} abre un minibuffer, escriben una
    palabra que quieren reemplazar, dan \texttt{enter}, escriben la palabra por
    la cual reemplazar y dan \texttt{enter}, para cada aparición de la palabra a
    reemplazar va a preguntar si la quieren reemplazar, presionan \texttt{y}
    para aceptar, \texttt{n} para cancelar, en ambos casos, va a la siguiente
    aparición de la palabra.
  \end{itemize}
\end{itemize}

\end{document}










